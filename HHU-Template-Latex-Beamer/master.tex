\pdfoptionpdfminorversion=5
\documentclass[9pt,hyperref={pdfpagelabels=false}]{beamer}

\mode<presentation> {
    \usetheme{HHUD}
    \setbeamercovered{invisible}
}
\usepackage[ngerman]{babel}
\usepackage[utf8x]{inputenc}
\usepackage{times}
\usepackage[T1]{fontenc}
\usepackage{amsmath}
\usepackage{subfigure}
\usepackage{graphicx}
\usepackage{hyperref}
\usepackage{xmpmulti}
\usepackage{multicol}
\usepackage{appendixnumberbeamer}

% background image
\usebackgroundtemplate{\includegraphics[width=\paperwidth]{fig/background}}
% commands for low and high decoration in frame foot
\newcommand{\footdecorationlow}{\usebackgroundtemplate{\includegraphics[width=\paperwidth]{fig/background_small}}}
\newcommand{\footdecorationhigh}{\usebackgroundtemplate{\includegraphics[width=\paperwidth]{fig/background}}}

% Fix build errors on debian (http://bugs.debian.org/cgi-bin/bugreport.cgi?bug=452333)
\providecommand \thispdfpagelabel[1]{} {}

%% Die folgenden Zeilen können auskommentiert werden, um vor jedem Kapitel eine Gliederungsfolie einzufügen
% \AtBeginSection[] {
%   \footdecorationhigh
%   \begin{frame}<beamer>
%     \thispagestyle{empty}
%     \frametitle{Gliederung}
%     \vspace{-5mm}
%     \tableofcontents[currentsection]
%   \end{frame}
%   \footdecorationlow
% }

% % % % % % % % % %  CHANGE TOPIC AND AUTHOR INFORMATION HERE % % % % % % % % %
\newcommand{\abschluss}{}                              % HIER UNZUTREFFENDES LÖSCHEN
\title{\abschluss{}\\Java is not just an island, it's dynamic!}                      % HIER DEN TITEL DER ARBEIT EINTRAGEN
\author{Christopher Probst, \\Matthias Hesse}                                                       % HIER DEN NAMEN UND VORNAMEN EINTRAGEN
\date{22.04.2015}                                                                % HIER DAS PRÄSENTATIONSDATUM EINTRAGEN
% % % % % % % % % % % % % % % % % % % % % % % % % % % % % % % % % % % % % % % %
\institute{Institut für Informatik\\Heinrich-Heine-Universität Düsseldorf}
\subject{Informatik}

%
% Hier beginnt das Dokument
%
\begin{document}

  \footdecorationhigh
  \begin{frame}
    \thispagestyle{empty}
    \titlepage
  \end{frame}

  \begin{frame}
    \thispagestyle{empty}
    \frametitle{Content}
    \vspace{-5mm}
    \tableofcontents
  \end{frame}
  \footdecorationlow

  % % % % % % % % % % Ab hier werden die LaTeX-Dateien der einzelnen Abschnitte eingefügt % % % % % % % % % %

  %!TEX root = master.tex

\section{What's Java ?}



\begin{frame}
  \frametitle{What's Java ?}
  \begin{itemize}
    \item Created by Mr. Gosling in 1995 (Sun Microsystems, later Oracle)
    \begin{center}
      \includegraphics[width=0.5\textwidth]{fig/young}
      \includegraphics[width=0.5\textwidth]{fig/old}
    \end{center}

    \item Typing: strong (no implicit casting), static (variables have types)
    \item Cross-platform (kind of...)
    \item Paradigms: object-oriented, structured, imperative, functional, generic, reflective, concurrent
  \end{itemize}
\end{frame}





  %!TEX root = master.tex

\section{Language features}


\begin{frame}
  \frametitle{java.lang.Object}

  \begin{itemize}
    \item Java is object-oriented, so everything is/should be an object!
  \end{itemize}
  \vspace{1cm}
  \begin{center}
    \includegraphics[width=0.4\textwidth]{fig/objects}
  \end{center}
\end{frame}


\begin{frame}
  \frametitle{java.lang.Object}

  \begin{itemize}
    \item Except byte, boolean, short, char, int, long, float, double
    \vspace{1cm}
    \item They are primitive (of course...)
  \end{itemize}
\end{frame}


\begin{frame}
  \frametitle{java.lang.generics.*}
  Java has generics:
  \vspace{0.25cm}
  \begin{itemize}
    \item Useful to write generic algorithms
    \vspace{0.4cm}
    \item Implement container classes (lists, maps, etc.)   
    \vspace{0.4cm}
    \item Generic compile errors can be hard to solve, but not as bad as C++
  \end{itemize}
\end{frame}


\begin{frame}
  \frametitle{java.lang.generics.*}
  Drawback: Only works with reference types =(
\end{frame}


\begin{frame}
  \frametitle{java.lang.generics.*}
	\begin{center}
	    \includegraphics[width=0.8\textwidth]{fig/primarray}    
	\end{center}

	\begin{center}
	    \includegraphics[width=0.5\textwidth]{fig/primitives}    
	\end{center}
\end{frame}


\begin{frame}
  \frametitle{java.lang.generics.*}
 \begin{itemize}
    \item Might change with Java 9 (2016) or Java 10 (2018)
    \vspace{0.4cm}
    \item HotSpot JEPs \& JSRs: Project Valhalla, Project Panama
  \end{itemize}  
\end{frame}


\begin{frame}
  \frametitle{java.lang.*}

 \begin{itemize}
    \item Code can only live in Classes
    \vspace{0.3cm}
    \item Syntactically close to C/C++ (Braces, etc.)
    \vspace{0.3cm}
    \item Reflective
    \vspace{0.3cm}
    \item Built-in support for multithreading (synchronized keyword, etc.)
    \vspace{0.3cm}
    \item Built-in support for serialization (transient keyword, etc.)
    \vspace{0.3cm}
    \item No multiple inheritance, all methods are virtual and overloadable
    \vspace{0.3cm}
    \item Since JDK8: Native support for lambda's and functional programming =)
  \end{itemize}  
  
\end{frame}


\begin{frame}
  \frametitle{java.util.*}

 \begin{itemize}
    \item Well defined concurrency libraries, arguably the best ones out there
    \vspace{0.3cm}
    \item Very performant network library implementation
    \vspace{0.3cm}  
    \item Therefore heavily used in backend systems (financial, social media, etc.)
    \vspace{0.3cm}
    \item Large number of open-source libraries
    \vspace{0.3cm}
    \item Gigantic community
  \end{itemize}  
  
\end{frame}


\begin{frame}
  \frametitle{java.lang.impl.HotSpot}
  \begin{itemize}
    \item Oracle's reference implementation is the HotSpot JVM
    \vspace{0.3cm}
    \item ByteCode interpreter
    \vspace{0.3cm}    
    \item JIT	(state-of-the-art, gets much better in Java 9)
    \vspace{0.3cm}    
    \item GC 	(state-of-the-art, even concurrent)    
  \end{itemize}
\end{frame}

\begin{frame}
  \frametitle{java.lang.impl.HotSpot}

  Runtime characteristics:
  \vspace{0.25cm}    
  \begin{itemize}
    \item It takes a while to warm up the JIT
    \vspace{0.3cm}    
    \item Learns about code-flow, dead-code elimination
    \vspace{0.3cm}    
    \item Dynamic stack allocation (through escape-analysis)
    \vspace{0.3cm}    
    \item Dynamic code optimizations
    \vspace{0.3cm}
    \item At some point very, very fast
    \vspace{0.3cm}
    \item Relatively high memory usage
    \vspace{0.3cm}
    \item In some situations on-par with C/C++ (no flame-war intended)
    \vspace{0.3cm}
    \item Conclusion: Perfect for long running, backend server applications
  \end{itemize}
\end{frame}

  %!TEX root = master.tex

\section{\$ java -is-dynamic}

\begin{frame}
  \frametitle{\$ java -is-dynamic}

  But can Java be considered a dynamic language ? 
  \vspace{0.4cm}
  \begin{center}
  \includegraphics[width=0.7\textwidth]{fig/dynlang}    
  \end{center}
\end{frame}

\begin{frame}
  \frametitle{\$ java -is-dynamic}
  \center

  "The term dynamic programming language describes a class of programming languages that share a number of common runtime characteristics that are available in static languages only during compilation, if at all.[…]” 

  \vspace{0.3cm}

  “[…]These behaviors can include the ability to extend the currently running program […] even by modifying the internals of the language itself, all during program execution. While these behaviors can be emulated in almost any language […] such behaviors are integral, built-in characteristics of dynamic languages.”
  \vspace{0.3cm}

T. Mikkonen and A. Taivalsaari,
“Using JavaScript as a real programming language” 2007.

\end{frame}

\begin{frame}
  \frametitle{\$ java -is-dynamic}
  \center
  So, can Java \frqq extend a currently running program\flqq, maybe \frqq even by modifying the internals of the language itself, all during program execution\flqq ?

  \vspace{0.25cm}
  \begin{itemize}
    \item Java reflection can modify certain aspects during runtime
    \vspace{0.4cm} 
    \item The Java class loader can load source code during runtime
    \vspace{0.4cm}      
    \item Javassist (Library) can create/modify Java classes during runtime
    \vspace{0.4cm}    
    \item But: It's a workaround (a hack)
    \vspace{0.4cm}
    \item So it's not an \frqq integral, built-in characteristic\flqq of Java
  \end{itemize}
\end{frame}

\begin{frame}
  \frametitle{\$ java -is-dynamic}

  Dynamic check list
  \vspace{0.25cm}
  \begin{itemize}
    \item Interactive (JavaREPL, wait for demo)
    \vspace{0.4cm} 
    \item Everything is an object (almost)
    \vspace{0.4cm}      
    \item Dynamic Typing (Simulated by using Object, wait for demo)
    \vspace{0.4cm}    
    \item Most things changeable at run-time (Well, it's hacky)
    \vspace{0.4cm}
    \item Reflection (Yes!)
    \vspace{0.4cm}
    \item Late-Bound Everything (Simulated by using Object, wait for demo)
    \vspace{0.4cm}
    \item Garbage Collected (Yes!)
    \vspace{0.4cm}
    \item Interpreted (Yes!)
  \end{itemize}
\end{frame}

\begin{frame}
  \frametitle{\$ java -conclusion}

  \vspace{0.25cm}
  \begin{itemize}
    \item Java is almost a dynamic language
    \vspace{0.4cm} 
    \item It can be stretched alot, but it hurts
    \vspace{0.4cm}      
    \item Not meant to be used this way
    \vspace{0.4cm}    
    \item Not necessarily a bad thing
    \vspace{0.4cm}
    \item Java sits between Python and C++
    \vspace{0.4cm}
    \item However, a lot of dynamic languages compile down to Java byte code!
    \vspace{0.4cm}
    \item Clojure, Jython, JRuby, and A LOT of others
  \end{itemize}
\end{frame}

  % % % % % % % % % % Ende der eingefügten LaTeX-Dateien % % % % % % % % % %

\end{document}

%
% Hier endet das Dokument
%